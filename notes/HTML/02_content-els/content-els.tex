\begin{description}
	\item[h1 - h6] \texttt{<h3>My Heading</h3>} Headings for different parts of the webpage
	\item[p] \texttt{<p>My sentence here</p>} For paragraphs and lines of text
	\item[ol] \texttt{<ol>...</ol>} Ordered list (numbered)
	\item[ul] \texttt{<ul>...</ul>} Unordered list (numbered)
	\item[li] \texttt{<li>List item content</li>} List item, only direct children of \texttt{ol} and \texttt{ul}
	\item[dl] \texttt{<dl>...</dl>} Description list
	\item[dt] \texttt{<dt>Term here</dt>} Description term
	\item[dd] \texttt{<dd>Definition here</dd>} Definition
	\item[a] \texttt{<a href="http://webaddress.com">Link Text</a>} Anchor or link; takes you somewhere \& needs \texttt{href} attribute
	\item[button] \texttt{<button>Do something</button>} Button for interactivity
	\item[img] \texttt{<img src="example.png" alt="A photo of a girl holding a book" />} Image. Don't forget the attributes.
	\item[blockquote] \texttt{<blockquote cite="">...</blockquote>} To add a quote, optional cite attribute
	\item[cite] \texttt{<cite>...</cite>} Citation
	\item[details] \texttt{<details>...</details>} Details
	\item[summary] \texttt{<summary>...</summary>} Summary for details (this is what is shown)
\end{description}

\href{https://css-tricks.com/a-complete-guide-to-links-and-buttons/}{The difference between a button and a link}

\begin{infobox}{Lorem Ipsum}
    \textbf{Lorem Ipsum} is latin. We use blocks of latin text as \textit{placeholders} when we are building websites, before real content is added. It's a good language as it doesn't mean much to us, but is still representative of how westernised languages would appear in blocks \& sentences.
    To get \textbf{Lorem Ipsum} you can type \textit{Lorem} followed by enter into vscode, or find a \href{https://www.lipsum.com/}{generator online here}.
\end{infobox}

\subsubsection{Tables}

Tables contain a lot of HTML. Below is an example of a simple table. We only use them in HTML for data which is good to represent as a table, however they are still used to layout emails

\begin{minted}{html}
<table>
	<tr>
		<th>Dessert</th>
		<th>Calories</th>
		<th>Fat</th>
		<th>Carbs</th>
	</tr>
	<tr>
		<td>Frozen yogurt</td>
		<td>159</td>
		<td>6.0</td>
		<td>24</td>
	</tr>
	<tr>
		<td>Ice cream sandwich</td>
		<td>237</td>
		<td>9.0</td>
		<td>37</td>
	</tr>
	<tr>
		<td>Eclair</td>
		<td>262</td>
		<td>16.0</td>
		<td>24</td>
	</tr>
</table>
\end{minted}
