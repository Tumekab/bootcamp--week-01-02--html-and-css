\section{A Section}

This is a paragraph. You need to end each one with two backslashes on the following line, otherwise you won't get a line break. I can't remember why I set it up like that.
\\

You can do \textbf{bold} text and \textit{italic} text. You can also do \texttt{monospace} text.
\\

If you want to but smart quotes around something you can use ``two backticks to start and two apostrophes to finish''.
\\

If you want to include a big block of code then you use the \texttt{minted} command:

% second curlies say what language to use for syntax highlighting
% you should probably indent everything inside a minted section
\begin{minted}{js}
    let x = 10;
    const z = n => n * n;

    z(10); // blah blah blah
\end{minted}

Sometimes you'll want to put code in a separate file. This will automatically link to the file on GitHub as copying from a PDF is a bit awful.

% \inputminted{js}{resources/example.js}

% sometimes page breaks are necessary
\pagebreak

There's a special one for PHP, which starts on line 3 to remove the \texttt{<?php} tag from the top:

% \phpinputminted{resources/example}

\subsection{A Sub Section}

You can also do sub-sections. Sub-sections will be included in the table of contents. And sub-sub-sections.

\subsubsection{A Sub Sub Section}

See. These won't be in the table of contents.
\\

I've also included a horizontal rule if you need a break of some sort.

\hr

Have a list:

\begin{itemize}
    \item Blah
    \item \textit{Blah}
    \item \textbf{Blah}
\end{itemize}

And a numbered one:

\begin{enumerate}
    \item Blah
    \item \textit{Blah}
    \item \textbf{Blah}
\end{enumerate}

\href{https://duckduckgo.com}{Links} can be useful. As can footnotes\footnote{Look at me, I'm a footnote!}
\\

\pagebreak

Sometimes it's useful to pull out some information:

\begin{infobox}{I'm an Infobox}
    Here's some side information about a thing. Maybe some historical context or a rant.
\end{infobox}

\quoteinline{Sometimes it's nice to include a quote from someone}{Someone, 1985}

It's also possible to include images:

% width first, then filename, then bottom margin, then caption text
% you can use most image types - vector based ones based for diagrams
\img{12cm}{resources/diagram.eps}{1em}{A very handy caption}
