\begin{description}
	\item[strong] \texttt{<strong>This is important</strong>} Important text, usually bold style
	\item[em] \texttt{<em>This is emphasised</em>} Emphasised text, usually italic style
	\item[b] \texttt{<b>This is bold</b>} Bold style
	\item[i] \texttt{<i>This is italic</i>} Itlaic style
	\item[span] \texttt{<span>This text has an element around it</span>} Like a div but for text
	\item[sup \& sub] \texttt{<sup>...</sup>} \texttt{<sub>...</sub>} Smaller raised or lowered text
	\item[time] \texttt{<time datetime="2020-07-21">21st July</time>} Any time or date.
	\item[q] \texttt{<q>Someone said this</q>} A quote but inline with the text
	\item[code] \texttt{<code>Inline</code>} Describes code
	\item[pre] \texttt{<pre>...</pre>} Used for code blocks
\end{description}

\begin{infobox}{Semantics}
    \textit{Semantics} is a word used to describe giving something meaning. We use it a lot when talking about HTML as that is what the elements are for. To give the content meaning.
\end{infobox}


