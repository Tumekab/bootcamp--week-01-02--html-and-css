\section{Your Best Friend}

DevTools are a developer’s best friend. Being able to inspect and debug code is a \textbf{neccessary} skill when it comes to web development.
\\

% \img{12cm}{tools.png}{1em}{Chrome DevTools}

To open DevTools you can either right click in the browser and click 'Inspect Element', or you can use a shortcut - CMD + OPTION + I (Chrome and Firefox), this will vary between browsers. You should ALWAYS have your DevTools open when coding.
\\

Essential Tabs for Front-end development in Chrome:
\\

\begin{itemize}
    \item Elements - most important tab for this course
    \item Network
    \item Audits
\end{itemize}

\subsection{Debugging}

The Elements tab alongside the Inspector and Device testing tools, are essential when marking up and styling web pages. They will give you:
\\

\begin{itemize}
    \item A built-in text editor, with immediate feedback in the browser
    \item Insights into the CSS cascade
    \item More visual tooling when working with the Box Model, CSS Grid etc
    \item Multiple screen widths, orientation and constraints to test against
\end{itemize}

This is not an exhaustive list.
\\

\subsection{Auditing}

The Network and Audit tabs evaluate your sites performance in terms of: 
\\

\begin{itemize}
    \item Accessibility
    \item Performance
    \item SEO
    \item Best Practices
\end{itemize}

This can be useful when developing your site for the first time and when looking at optimisations.
\\