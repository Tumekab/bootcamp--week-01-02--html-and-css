A wirefrmae is a representation of what is there and how it works.

Crucially, a way of thinking about the content and behaviour \textbf{before visual design is considered.}

That’s why wireframes will look \textbf{bland and grey.}

We want to \textbf{focus on the function} of the site before we start thinking about style.

It will be given to:

\begin{itemize}
    \item clients.
    \item designers.
    \item developers \& sys admins.
\end{itemize}

\begin{infobox}{Educate the client}
    Clients can sometimes be unfamiliar with digital processes. When sending a wireframe link to a client you should include an explanation of what a wireframe is for, what feedback is useful, and what isn’t.
    Clients will circulate a wireframe link within the organisation, often without your helpful explanation of what a wireframe is for so add a note to the wireframe saying what it is.
\end{infobox}

\subsubsection{Wireframe Software}

\begin{itemize}
    \item Balsamiq: \href{https://balsamiq.com/products/}{https://balsamiq.com/products/}
    \item Moqups: \href{https://moqups.com/}{https://moqups.com/}
    \item UXPin: \href{https://www.uxpin.com/}{https://www.uxpin.com/}
    \item AdobeXD: \href{https://www.adobe.com/uk/products/xd.html}{https://www.adobe.com/uk/products/xd.html}
    \item \href{List on Creative Bloc here}{https://www.creativebloq.com/wireframes/top-wireframing-tools-11121302}
    \item Any design software too
\end{itemize}

\begin{infobox}{Prototypes}
    Wireframing is good but it doesn't give you interactivity. You can build a prototype to show that however. CSS Frameworks (see css.pdf) can be useful for this.
\end{infobox}

