When you're viewing a website on the internet you're asking for a file via the url. A \textit{server} sends you this file in response to this request, which the \textit{browser} then interprets or renders for you.

Programming for the server is generally referred to as \textit{backend}, and programming for the browser or client is usually referred to as \textit{frontend}.

Here are some languages you might hear of, this is not a definitive list.

\subsubsection{Backend Languages}

\begin{itemize}
    \item Javascript
    \item PHP
    \item Python
    \item Ruby
    \item Java
    \item C / C\# / C++
    \item .net
\end{itemize}

\subsubsection{Frontend Languages}

\begin{itemize}
    \item HTML
    \item CSS
    \item Javascript
\end{itemize}

These define roles you will hear of in companies. Here is a list, again this is not definitive:

\subsubsection{Job Roles}

\begin{itemize}
    \item UX
    \item Designer / UI
    \item Front End Developers (Front End Designers, UI Developers)
    \item Back End Developers
    \item System Administrators
    \item Product Owners
    \item Project Managers
    \item Clients
    \item Testers
\end{itemize}

