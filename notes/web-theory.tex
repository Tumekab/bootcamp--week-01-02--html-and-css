% setup the document
\documentclass[b5paper,openany]{book}

% setup variables ------------------------------

% your name
\newcommand\instructor{Ruth John}

% the week number
\newcommand\weekno{1}

% week main title
\newcommand\maintitle{Web Theory}

% week subtitle
\newcommand\subtitle{General information \& guides about building things for the web.}

% link to notes - relative to github.com/develop-me/
\newcommand\github{bootcamp--week--01-02--html-and-css/blob/master/notes/}

% setup functions, styling, etc.
\input{../../latex-templates/template.tex}

% the structure of the document
\begin{document}

% render the title page (uses variables above)
\tp

\quotepage{A flower does not think about competing with the flower next to it. It just blooms.}{Anonymous}

\tableofcontents

\input{../../latex-templates/preface.tex}

\chapter{General}
\section{A Section}

This is a paragraph. You need to end each one with two backslashes on the following line, otherwise you won't get a line break. I can't remember why I set it up like that.
\\

You can do \textbf{bold} text and \textit{italic} text. You can also do \texttt{monospace} text.
\\

If you want to but smart quotes around something you can use ``two backticks to start and two apostrophes to finish''.
\\

If you want to include a big block of code then you use the \texttt{minted} command:

% second curlies say what language to use for syntax highlighting
% you should probably indent everything inside a minted section
\begin{minted}{js}
    let x = 10;
    const z = n => n * n;

    z(10); // blah blah blah
\end{minted}

Sometimes you'll want to put code in a separate file. This will automatically link to the file on GitHub as copying from a PDF is a bit awful.

% \inputminted{js}{resources/example.js}

% sometimes page breaks are necessary
\pagebreak

There's a special one for PHP, which starts on line 3 to remove the \texttt{<?php} tag from the top:

% \phpinputminted{resources/example}

\subsection{A Sub Section}

You can also do sub-sections. Sub-sections will be included in the table of contents. And sub-sub-sections.

\subsubsection{A Sub Sub Section}

See. These won't be in the table of contents.
\\

I've also included a horizontal rule if you need a break of some sort.

\hr

Have a list:

\begin{itemize}
    \item Blah
    \item \textit{Blah}
    \item \textbf{Blah}
\end{itemize}

And a numbered one:

\begin{enumerate}
    \item Blah
    \item \textit{Blah}
    \item \textbf{Blah}
\end{enumerate}

\href{https://duckduckgo.com}{Links} can be useful. As can footnotes\footnote{Look at me, I'm a footnote!}
\\

\pagebreak

Sometimes it's useful to pull out some information:

\begin{infobox}{I'm an Infobox}
    Here's some side information about a thing. Maybe some historical context or a rant.
\end{infobox}

\quoteinline{Sometimes it's nice to include a quote from someone}{Someone, 1985}

It's also possible to include images:

% width first, then filename, then bottom margin, then caption text
% you can use most image types - vector based ones based for diagrams
\img{12cm}{resources/diagram.eps}{1em}{A very handy caption}


\chapter{Assistive Techonologies}
\section{A Section}

This is a paragraph. You need to end each one with two backslashes on the following line, otherwise you won't get a line break. I can't remember why I set it up like that.
\\

You can do \textbf{bold} text and \textit{italic} text. You can also do \texttt{monospace} text.
\\

If you want to but smart quotes around something you can use ``two backticks to start and two apostrophes to finish''.
\\

If you want to include a big block of code then you use the \texttt{minted} command:

% second curlies say what language to use for syntax highlighting
% you should probably indent everything inside a minted section
\begin{minted}{js}
    let x = 10;
    const z = n => n * n;

    z(10); // blah blah blah
\end{minted}

Sometimes you'll want to put code in a separate file. This will automatically link to the file on GitHub as copying from a PDF is a bit awful.

% \inputminted{js}{resources/example.js}

% sometimes page breaks are necessary
\pagebreak

There's a special one for PHP, which starts on line 3 to remove the \texttt{<?php} tag from the top:

% \phpinputminted{resources/example}

\subsection{A Sub Section}

You can also do sub-sections. Sub-sections will be included in the table of contents. And sub-sub-sections.

\subsubsection{A Sub Sub Section}

See. These won't be in the table of contents.
\\

I've also included a horizontal rule if you need a break of some sort.

\hr

Have a list:

\begin{itemize}
    \item Blah
    \item \textit{Blah}
    \item \textbf{Blah}
\end{itemize}

And a numbered one:

\begin{enumerate}
    \item Blah
    \item \textit{Blah}
    \item \textbf{Blah}
\end{enumerate}

\href{https://duckduckgo.com}{Links} can be useful. As can footnotes\footnote{Look at me, I'm a footnote!}
\\

\pagebreak

Sometimes it's useful to pull out some information:

\begin{infobox}{I'm an Infobox}
    Here's some side information about a thing. Maybe some historical context or a rant.
\end{infobox}

\quoteinline{Sometimes it's nice to include a quote from someone}{Someone, 1985}

It's also possible to include images:

% width first, then filename, then bottom margin, then caption text
% you can use most image types - vector based ones based for diagrams
\img{12cm}{resources/diagram.eps}{1em}{A very handy caption}


\chapter{Devtools}
\section{Your Best Friend}

DevTools are a developer’s best friend. Being able to inspect and debug code is a \textbf{neccessary} skill when it comes to web development.
\\

% \img{12cm}{tools.png}{1em}{Chrome DevTools}

To open DevTools you can either right click in the browser and click 'Inspect Element', or you can use a shortcut - CMD + OPTION + I (Chrome and Firefox), this will vary between browsers. You should ALWAYS have your DevTools open when coding.
\\

\subsection{Essential Tabs}

\begin{itemize}
    \item Elements - most important tab for this course
    \item Network
    \item Audits
\end{itemize}

\subsection{Debugging}

The Elements tab alongside the Inspector and Device testing tools, are essential when marking up and styling web pages. They will give you:
\\

\begin{itemize}
    \item A built-in text editor, with immediate feedback in the browser
    \item Insights into the CSS cascade
    \item More visual tooling when working with the Box Model, CSS Grid etc
    \item Multiple screen widths, orientation and constraints to test against
\end{itemize}

This is not an exhaustive list.
\\

\subsection{Auditing}

The Network and Audit tabs evaluate your sites performance in terms of: 
\\

\begin{itemize}
    \item Accessibility
    \item Performance
    \item SEO
    \item Best Practices
\end{itemize}

This can be useful when developing your site for the first time and when looking at optimisations.
\\

\chapter{Requirement Gathering}
\section{A Section}

This is a paragraph. You need to end each one with two backslashes on the following line, otherwise you won't get a line break. I can't remember why I set it up like that.
\\

You can do \textbf{bold} text and \textit{italic} text. You can also do \texttt{monospace} text.
\\

If you want to but smart quotes around something you can use ``two backticks to start and two apostrophes to finish''.
\\

If you want to include a big block of code then you use the \texttt{minted} command:

% second curlies say what language to use for syntax highlighting
% you should probably indent everything inside a minted section
\begin{minted}{js}
    let x = 10;
    const z = n => n * n;

    z(10); // blah blah blah
\end{minted}

Sometimes you'll want to put code in a separate file. This will automatically link to the file on GitHub as copying from a PDF is a bit awful.

% \inputminted{js}{resources/example.js}

% sometimes page breaks are necessary
\pagebreak

There's a special one for PHP, which starts on line 3 to remove the \texttt{<?php} tag from the top:

% \phpinputminted{resources/example}

\subsection{A Sub Section}

You can also do sub-sections. Sub-sections will be included in the table of contents. And sub-sub-sections.

\subsubsection{A Sub Sub Section}

See. These won't be in the table of contents.
\\

I've also included a horizontal rule if you need a break of some sort.

\hr

Have a list:

\begin{itemize}
    \item Blah
    \item \textit{Blah}
    \item \textbf{Blah}
\end{itemize}

And a numbered one:

\begin{enumerate}
    \item Blah
    \item \textit{Blah}
    \item \textbf{Blah}
\end{enumerate}

\href{https://duckduckgo.com}{Links} can be useful. As can footnotes\footnote{Look at me, I'm a footnote!}
\\

\pagebreak

Sometimes it's useful to pull out some information:

\begin{infobox}{I'm an Infobox}
    Here's some side information about a thing. Maybe some historical context or a rant.
\end{infobox}

\quoteinline{Sometimes it's nice to include a quote from someone}{Someone, 1985}

It's also possible to include images:

% width first, then filename, then bottom margin, then caption text
% you can use most image types - vector based ones based for diagrams
\img{12cm}{resources/diagram.eps}{1em}{A very handy caption}


\chapter{Sitemapping}
\section{A Section}

This is a paragraph. You need to end each one with two backslashes on the following line, otherwise you won't get a line break. I can't remember why I set it up like that.
\\

You can do \textbf{bold} text and \textit{italic} text. You can also do \texttt{monospace} text.
\\

If you want to but smart quotes around something you can use ``two backticks to start and two apostrophes to finish''.
\\

If you want to include a big block of code then you use the \texttt{minted} command:

% second curlies say what language to use for syntax highlighting
% you should probably indent everything inside a minted section
\begin{minted}{js}
    let x = 10;
    const z = n => n * n;

    z(10); // blah blah blah
\end{minted}

Sometimes you'll want to put code in a separate file. This will automatically link to the file on GitHub as copying from a PDF is a bit awful.

% \inputminted{js}{resources/example.js}

% sometimes page breaks are necessary
\pagebreak

There's a special one for PHP, which starts on line 3 to remove the \texttt{<?php} tag from the top:

% \phpinputminted{resources/example}

\subsection{A Sub Section}

You can also do sub-sections. Sub-sections will be included in the table of contents. And sub-sub-sections.

\subsubsection{A Sub Sub Section}

See. These won't be in the table of contents.
\\

I've also included a horizontal rule if you need a break of some sort.

\hr

Have a list:

\begin{itemize}
    \item Blah
    \item \textit{Blah}
    \item \textbf{Blah}
\end{itemize}

And a numbered one:

\begin{enumerate}
    \item Blah
    \item \textit{Blah}
    \item \textbf{Blah}
\end{enumerate}

\href{https://duckduckgo.com}{Links} can be useful. As can footnotes\footnote{Look at me, I'm a footnote!}
\\

\pagebreak

Sometimes it's useful to pull out some information:

\begin{infobox}{I'm an Infobox}
    Here's some side information about a thing. Maybe some historical context or a rant.
\end{infobox}

\quoteinline{Sometimes it's nice to include a quote from someone}{Someone, 1985}

It's also possible to include images:

% width first, then filename, then bottom margin, then caption text
% you can use most image types - vector based ones based for diagrams
\img{12cm}{resources/diagram.eps}{1em}{A very handy caption}


\chapter{Scamping}
\section{A Section}

This is a paragraph. You need to end each one with two backslashes on the following line, otherwise you won't get a line break. I can't remember why I set it up like that.
\\

You can do \textbf{bold} text and \textit{italic} text. You can also do \texttt{monospace} text.
\\

If you want to but smart quotes around something you can use ``two backticks to start and two apostrophes to finish''.
\\

If you want to include a big block of code then you use the \texttt{minted} command:

% second curlies say what language to use for syntax highlighting
% you should probably indent everything inside a minted section
\begin{minted}{js}
    let x = 10;
    const z = n => n * n;

    z(10); // blah blah blah
\end{minted}

Sometimes you'll want to put code in a separate file. This will automatically link to the file on GitHub as copying from a PDF is a bit awful.

% \inputminted{js}{resources/example.js}

% sometimes page breaks are necessary
\pagebreak

There's a special one for PHP, which starts on line 3 to remove the \texttt{<?php} tag from the top:

% \phpinputminted{resources/example}

\subsection{A Sub Section}

You can also do sub-sections. Sub-sections will be included in the table of contents. And sub-sub-sections.

\subsubsection{A Sub Sub Section}

See. These won't be in the table of contents.
\\

I've also included a horizontal rule if you need a break of some sort.

\hr

Have a list:

\begin{itemize}
    \item Blah
    \item \textit{Blah}
    \item \textbf{Blah}
\end{itemize}

And a numbered one:

\begin{enumerate}
    \item Blah
    \item \textit{Blah}
    \item \textbf{Blah}
\end{enumerate}

\href{https://duckduckgo.com}{Links} can be useful. As can footnotes\footnote{Look at me, I'm a footnote!}
\\

\pagebreak

Sometimes it's useful to pull out some information:

\begin{infobox}{I'm an Infobox}
    Here's some side information about a thing. Maybe some historical context or a rant.
\end{infobox}

\quoteinline{Sometimes it's nice to include a quote from someone}{Someone, 1985}

It's also possible to include images:

% width first, then filename, then bottom margin, then caption text
% you can use most image types - vector based ones based for diagrams
\img{12cm}{resources/diagram.eps}{1em}{A very handy caption}


\chapter{Wireframing}
\section{A Section}

This is a paragraph. You need to end each one with two backslashes on the following line, otherwise you won't get a line break. I can't remember why I set it up like that.
\\

You can do \textbf{bold} text and \textit{italic} text. You can also do \texttt{monospace} text.
\\

If you want to but smart quotes around something you can use ``two backticks to start and two apostrophes to finish''.
\\

If you want to include a big block of code then you use the \texttt{minted} command:

% second curlies say what language to use for syntax highlighting
% you should probably indent everything inside a minted section
\begin{minted}{js}
    let x = 10;
    const z = n => n * n;

    z(10); // blah blah blah
\end{minted}

Sometimes you'll want to put code in a separate file. This will automatically link to the file on GitHub as copying from a PDF is a bit awful.

% \inputminted{js}{resources/example.js}

% sometimes page breaks are necessary
\pagebreak

There's a special one for PHP, which starts on line 3 to remove the \texttt{<?php} tag from the top:

% \phpinputminted{resources/example}

\subsection{A Sub Section}

You can also do sub-sections. Sub-sections will be included in the table of contents. And sub-sub-sections.

\subsubsection{A Sub Sub Section}

See. These won't be in the table of contents.
\\

I've also included a horizontal rule if you need a break of some sort.

\hr

Have a list:

\begin{itemize}
    \item Blah
    \item \textit{Blah}
    \item \textbf{Blah}
\end{itemize}

And a numbered one:

\begin{enumerate}
    \item Blah
    \item \textit{Blah}
    \item \textbf{Blah}
\end{enumerate}

\href{https://duckduckgo.com}{Links} can be useful. As can footnotes\footnote{Look at me, I'm a footnote!}
\\

\pagebreak

Sometimes it's useful to pull out some information:

\begin{infobox}{I'm an Infobox}
    Here's some side information about a thing. Maybe some historical context or a rant.
\end{infobox}

\quoteinline{Sometimes it's nice to include a quote from someone}{Someone, 1985}

It's also possible to include images:

% width first, then filename, then bottom margin, then caption text
% you can use most image types - vector based ones based for diagrams
\img{12cm}{resources/diagram.eps}{1em}{A very handy caption}


\chapter{Design Handover}
\section{A Section}

This is a paragraph. You need to end each one with two backslashes on the following line, otherwise you won't get a line break. I can't remember why I set it up like that.
\\

You can do \textbf{bold} text and \textit{italic} text. You can also do \texttt{monospace} text.
\\

If you want to but smart quotes around something you can use ``two backticks to start and two apostrophes to finish''.
\\

If you want to include a big block of code then you use the \texttt{minted} command:

% second curlies say what language to use for syntax highlighting
% you should probably indent everything inside a minted section
\begin{minted}{js}
    let x = 10;
    const z = n => n * n;

    z(10); // blah blah blah
\end{minted}

Sometimes you'll want to put code in a separate file. This will automatically link to the file on GitHub as copying from a PDF is a bit awful.

% \inputminted{js}{resources/example.js}

% sometimes page breaks are necessary
\pagebreak

There's a special one for PHP, which starts on line 3 to remove the \texttt{<?php} tag from the top:

% \phpinputminted{resources/example}

\subsection{A Sub Section}

You can also do sub-sections. Sub-sections will be included in the table of contents. And sub-sub-sections.

\subsubsection{A Sub Sub Section}

See. These won't be in the table of contents.
\\

I've also included a horizontal rule if you need a break of some sort.

\hr

Have a list:

\begin{itemize}
    \item Blah
    \item \textit{Blah}
    \item \textbf{Blah}
\end{itemize}

And a numbered one:

\begin{enumerate}
    \item Blah
    \item \textit{Blah}
    \item \textbf{Blah}
\end{enumerate}

\href{https://duckduckgo.com}{Links} can be useful. As can footnotes\footnote{Look at me, I'm a footnote!}
\\

\pagebreak

Sometimes it's useful to pull out some information:

\begin{infobox}{I'm an Infobox}
    Here's some side information about a thing. Maybe some historical context or a rant.
\end{infobox}

\quoteinline{Sometimes it's nice to include a quote from someone}{Someone, 1985}

It's also possible to include images:

% width first, then filename, then bottom margin, then caption text
% you can use most image types - vector based ones based for diagrams
\img{12cm}{resources/diagram.eps}{1em}{A very handy caption}


\chapter{Design Breakdown}
\section{A Section}

This is a paragraph. You need to end each one with two backslashes on the following line, otherwise you won't get a line break. I can't remember why I set it up like that.
\\

You can do \textbf{bold} text and \textit{italic} text. You can also do \texttt{monospace} text.
\\

If you want to but smart quotes around something you can use ``two backticks to start and two apostrophes to finish''.
\\

If you want to include a big block of code then you use the \texttt{minted} command:

% second curlies say what language to use for syntax highlighting
% you should probably indent everything inside a minted section
\begin{minted}{js}
    let x = 10;
    const z = n => n * n;

    z(10); // blah blah blah
\end{minted}

Sometimes you'll want to put code in a separate file. This will automatically link to the file on GitHub as copying from a PDF is a bit awful.

% \inputminted{js}{resources/example.js}

% sometimes page breaks are necessary
\pagebreak

There's a special one for PHP, which starts on line 3 to remove the \texttt{<?php} tag from the top:

% \phpinputminted{resources/example}

\subsection{A Sub Section}

You can also do sub-sections. Sub-sections will be included in the table of contents. And sub-sub-sections.

\subsubsection{A Sub Sub Section}

See. These won't be in the table of contents.
\\

I've also included a horizontal rule if you need a break of some sort.

\hr

Have a list:

\begin{itemize}
    \item Blah
    \item \textit{Blah}
    \item \textbf{Blah}
\end{itemize}

And a numbered one:

\begin{enumerate}
    \item Blah
    \item \textit{Blah}
    \item \textbf{Blah}
\end{enumerate}

\href{https://duckduckgo.com}{Links} can be useful. As can footnotes\footnote{Look at me, I'm a footnote!}
\\

\pagebreak

Sometimes it's useful to pull out some information:

\begin{infobox}{I'm an Infobox}
    Here's some side information about a thing. Maybe some historical context or a rant.
\end{infobox}

\quoteinline{Sometimes it's nice to include a quote from someone}{Someone, 1985}

It's also possible to include images:

% width first, then filename, then bottom margin, then caption text
% you can use most image types - vector based ones based for diagrams
\img{12cm}{resources/diagram.eps}{1em}{A very handy caption}


\chapter{Accessibility}
\section{A Section}

This is a paragraph. You need to end each one with two backslashes on the following line, otherwise you won't get a line break. I can't remember why I set it up like that.
\\

You can do \textbf{bold} text and \textit{italic} text. You can also do \texttt{monospace} text.
\\

If you want to but smart quotes around something you can use ``two backticks to start and two apostrophes to finish''.
\\

If you want to include a big block of code then you use the \texttt{minted} command:

% second curlies say what language to use for syntax highlighting
% you should probably indent everything inside a minted section
\begin{minted}{js}
    let x = 10;
    const z = n => n * n;

    z(10); // blah blah blah
\end{minted}

Sometimes you'll want to put code in a separate file. This will automatically link to the file on GitHub as copying from a PDF is a bit awful.

% \inputminted{js}{resources/example.js}

% sometimes page breaks are necessary
\pagebreak

There's a special one for PHP, which starts on line 3 to remove the \texttt{<?php} tag from the top:

% \phpinputminted{resources/example}

\subsection{A Sub Section}

You can also do sub-sections. Sub-sections will be included in the table of contents. And sub-sub-sections.

\subsubsection{A Sub Sub Section}

See. These won't be in the table of contents.
\\

I've also included a horizontal rule if you need a break of some sort.

\hr

Have a list:

\begin{itemize}
    \item Blah
    \item \textit{Blah}
    \item \textbf{Blah}
\end{itemize}

And a numbered one:

\begin{enumerate}
    \item Blah
    \item \textit{Blah}
    \item \textbf{Blah}
\end{enumerate}

\href{https://duckduckgo.com}{Links} can be useful. As can footnotes\footnote{Look at me, I'm a footnote!}
\\

\pagebreak

Sometimes it's useful to pull out some information:

\begin{infobox}{I'm an Infobox}
    Here's some side information about a thing. Maybe some historical context or a rant.
\end{infobox}

\quoteinline{Sometimes it's nice to include a quote from someone}{Someone, 1985}

It's also possible to include images:

% width first, then filename, then bottom margin, then caption text
% you can use most image types - vector based ones based for diagrams
\img{12cm}{resources/diagram.eps}{1em}{A very handy caption}


\chapter{Upload to a server}
\section{A Section}

This is a paragraph. You need to end each one with two backslashes on the following line, otherwise you won't get a line break. I can't remember why I set it up like that.
\\

You can do \textbf{bold} text and \textit{italic} text. You can also do \texttt{monospace} text.
\\

If you want to but smart quotes around something you can use ``two backticks to start and two apostrophes to finish''.
\\

If you want to include a big block of code then you use the \texttt{minted} command:

% second curlies say what language to use for syntax highlighting
% you should probably indent everything inside a minted section
\begin{minted}{js}
    let x = 10;
    const z = n => n * n;

    z(10); // blah blah blah
\end{minted}

Sometimes you'll want to put code in a separate file. This will automatically link to the file on GitHub as copying from a PDF is a bit awful.

% \inputminted{js}{resources/example.js}

% sometimes page breaks are necessary
\pagebreak

There's a special one for PHP, which starts on line 3 to remove the \texttt{<?php} tag from the top:

% \phpinputminted{resources/example}

\subsection{A Sub Section}

You can also do sub-sections. Sub-sections will be included in the table of contents. And sub-sub-sections.

\subsubsection{A Sub Sub Section}

See. These won't be in the table of contents.
\\

I've also included a horizontal rule if you need a break of some sort.

\hr

Have a list:

\begin{itemize}
    \item Blah
    \item \textit{Blah}
    \item \textbf{Blah}
\end{itemize}

And a numbered one:

\begin{enumerate}
    \item Blah
    \item \textit{Blah}
    \item \textbf{Blah}
\end{enumerate}

\href{https://duckduckgo.com}{Links} can be useful. As can footnotes\footnote{Look at me, I'm a footnote!}
\\

\pagebreak

Sometimes it's useful to pull out some information:

\begin{infobox}{I'm an Infobox}
    Here's some side information about a thing. Maybe some historical context or a rant.
\end{infobox}

\quoteinline{Sometimes it's nice to include a quote from someone}{Someone, 1985}

It's also possible to include images:

% width first, then filename, then bottom margin, then caption text
% you can use most image types - vector based ones based for diagrams
\img{12cm}{resources/diagram.eps}{1em}{A very handy caption}


\chapter{Testing your website}
\section{A Section}

This is a paragraph. You need to end each one with two backslashes on the following line, otherwise you won't get a line break. I can't remember why I set it up like that.
\\

You can do \textbf{bold} text and \textit{italic} text. You can also do \texttt{monospace} text.
\\

If you want to but smart quotes around something you can use ``two backticks to start and two apostrophes to finish''.
\\

If you want to include a big block of code then you use the \texttt{minted} command:

% second curlies say what language to use for syntax highlighting
% you should probably indent everything inside a minted section
\begin{minted}{js}
    let x = 10;
    const z = n => n * n;

    z(10); // blah blah blah
\end{minted}

Sometimes you'll want to put code in a separate file. This will automatically link to the file on GitHub as copying from a PDF is a bit awful.

% \inputminted{js}{resources/example.js}

% sometimes page breaks are necessary
\pagebreak

There's a special one for PHP, which starts on line 3 to remove the \texttt{<?php} tag from the top:

% \phpinputminted{resources/example}

\subsection{A Sub Section}

You can also do sub-sections. Sub-sections will be included in the table of contents. And sub-sub-sections.

\subsubsection{A Sub Sub Section}

See. These won't be in the table of contents.
\\

I've also included a horizontal rule if you need a break of some sort.

\hr

Have a list:

\begin{itemize}
    \item Blah
    \item \textit{Blah}
    \item \textbf{Blah}
\end{itemize}

And a numbered one:

\begin{enumerate}
    \item Blah
    \item \textit{Blah}
    \item \textbf{Blah}
\end{enumerate}

\href{https://duckduckgo.com}{Links} can be useful. As can footnotes\footnote{Look at me, I'm a footnote!}
\\

\pagebreak

Sometimes it's useful to pull out some information:

\begin{infobox}{I'm an Infobox}
    Here's some side information about a thing. Maybe some historical context or a rant.
\end{infobox}

\quoteinline{Sometimes it's nice to include a quote from someone}{Someone, 1985}

It's also possible to include images:

% width first, then filename, then bottom margin, then caption text
% you can use most image types - vector based ones based for diagrams
\img{12cm}{resources/diagram.eps}{1em}{A very handy caption}


\nchapter{Glossary}
\begin{itemize}[leftmargin=*]
    \item
        \textbf{HTML}:
        Hypertext Markup Language. The language you write to describe content being displayed on a webpage
    \item
        \textbf{XML}:
        The language HTML is based on, it's a \textit{data structure}.
    \item
        \textbf{element}:
        Sometimes referred to as \textit{tag}. The part of the HTML inside angle brackets which describes the content.
    \item
        \textbf{attribute}:
        Gives the element more information. Write like \texttt{attribute="parameter"}. Add these to the opening element with spaces inbetween them.
    \item
        \textbf{nesting}:
        When we put elements inside of other elements
    \item
        \textbf{parent}:
        The containing element
    \item
        \textbf{child/children}:
        The element(s) inside other elements.
    \item
        \textbf{whitespace}:
        The parts of a file that don't contain anything. Like spaces, new lines or tabs.
    \item
        \textbf{comment}:
        Comments are parts of a code file that are ignored by the program running it.

\end{itemize}


\input{../../latex-templates/colophon.tex}

\end{document}