\subsubsection{Block Element Modifier}

BEM is a class naming convention. It may seem long winded, but we have already seen the benefits of the button class, you can use it on any element and it will adopt the styles.

It also helps other developers understand what CSS is doing what.

A component is a block and inside that you have elements. These are separated by two underscores. If there are modifications to the default layout or style, you add another term to any of these with a double hyphen.

\subsubsection{Example}

\begin{minted}{html}
<article class="card">

    <header class="card__head">
        <h2 class="card__head__text">Card heading</h2>
    </header>

    <figure class="card__img">
        <img src="" alt="" />
    </figure>

    <p class="card__blurb">Some text here</p>

    <a href="" class="card__link">A link</a>

</article>
\end{minted}

A card with no image could be described as `.card--noimage`

