Is used to distribute items along an axis .The browser is in control of the distribution

\subsubsection{Container Properties}

\begin{itemize}
    \item \texttt{display: flex;}
    \item \texttt{flex-direction} row | row-reverse | column | column-reverse
    \item \texttt{justify-content} flex-start | flex-end | center | space-around | space-between | space-evenly
    \item \texttt{flex-wrap} if you want to wrap
    \item \texttt{align-items} on cross axis;
    \item \texttt{align-content} cross axis space in container
\end{itemize}

\subsubsection{Child Properties}

\begin{itemize}
    \item \texttt{order} default 0, change to order
    \item \texttt{flex-grow} default 0, change to give item more room
    \item \texttt{flex-shrink} default 0, change to give item less room
\end{itemize}


\href{https://css-tricks.com/snippets/css/a-guide-to-flexbox/}{Flexbox on CSS Tricks}. This is an incredibly good guide. It is recommended over anything we can put in these notes. It is also the most hit page on the whole of the website, we all use it all the time!
