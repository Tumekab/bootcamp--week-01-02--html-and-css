Specificity determines, which CSS rule is applied by the browsers.
\\

It’s usually the reason why your CSS-rules don’t apply to some elements, although you think they should.

\subsubsection{Rules}

Every selector has its place in the specificity hierarchy.

1. Inline styles
2. IDs
3. Classes, attributes \& pseudo classes
4. Elements \& pseudo elements

If two selectors apply to the same element, the one with higher specificity wins.
\\

Order is important - latest rule still applies...Unless specificity value is higher

\begin{minted}{html}
<div id=”thing”>Content Here</div>
\end{minted}

\begin{minted}{css}
    #thing {background-color: red;}
    div {background-color: blue;}
\end{minted}

The background colour of the div will be red.


