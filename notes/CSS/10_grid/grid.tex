Designers have been using grids for a very long time in print - magazine and newspapers.
\\
This transferred over to web design. Specifically when we were working with fix width sites, we didn't have flexbox or grid back then.
\\
So we created our own grid systems: CSS you added to your website which helped put elements in the right place. They were usually 12 columns and you added classes to your HTML (a bit like bootstrap today, but just for layout).


\subsubsection{CSS Grid is here}

Bonuses include:
- Not tied into a system
- No addons needed
- No added classes


We use it for laying out a page, it's easy to put content in the right place. Think both horizontal and vertical axis, rather than flexbox's one. It's DOM dependent.


\subsection{How to use grid}

\begin{minted}{css}
.page__home {
    display: grid;
}
\end{minted}


\subsubsection{Specify the amount of cols \& rows}

\begin{minted}{css}
.page__home {
    display: grid;
    grid-template-rows: repeat(3, auto);
    grid-template-columns: 20vw 1fr 1fr 10vw;
}
\end{minted}


\subsubsection{How do we get things in places?}

We can name the areas we have created

\begin{minted}{css}
.page__home {
    display: grid;
    grid-template-rows: 20vh 1fr 1fr 16vh;
    grid-template-columns: repeat(3, 1fr);
    grid-template-areas:
        "top top top"
        "middle middle side"
        "middle middle ."
        ". bottom ."
    ;
}
\end{minted}

Then we can reference those names when we're styling the child sections

\begin{minted}{css}
.header-main {
    grid-area: top;
}
\end{minted}

\subsubsection{Or}

Grids have number lines automatically, we can use those

\begin{minted}{css}
.header-main {
    grid-row-start: 1;
    grid-row-end: 2;
    grid-column-start: 1;
    grid-column-end: 4;
}
\end{minted}

\subsubsection{Shorthand}

\begin{minted}{css}
.header-main {
    grid-row: 1 / 2;
    grid-column: 1 / 4;
}
\end{minted}

\begin{minted}{css}
.header-main {
    grid-area: 1 / 1 / 2 / 4;
}
\end{minted}


\subsubsection{Explicit and implicit grids}

Explicit is what you know, implicit is what you don't know. When we specify rows and columns and names as above we create an \textit{explicit} grid. If we didn't CSS grid would place items automatically for us and be an \textit{implicit} grid


\begin{minted}{css}
section {
    display: grid;
    /* how big the rows are */
    grid-auto-rows: 140px;
    /* Direction */
    grid-auto-flow: column;
}
\end{minted}


\subsubsection{The `fr` unit}

The `fr` unit is only available for grid
(We are considering it as a global unit \o/)

\subsubsection{Grid Gap}

Adds gaps in-between cells

\begin{minted}{css}
section {
    display: grid;
    grid-template-columns: repeat(3, 1fr);
    grid-column-gap: 1rem;
    grid-row-gap: 1rem;
    /* or */
    grid-gap: 1rem;
}
\end{minted}

\subsubsection{`minmax()`}

Like min/max width/height. Specifies how big or small something can be

\begin{minted}{css}
section {
    display: grid;
    grid-template-row: minmax(100px, auto);
}
\end{minted}


\subsubsection{You can name grid lines}

\begin{minted}{css}
section {
    display: grid;
    grid-template-columns: [first] 40px [line2] 50px [line3] auto [col4-start] 50px [five] 40px [end];
}
\end{minted}


\subsection{Children}

Alignment

\begin{minted}{css}
/* on parent */
section {
    justify-items: start;
    align-items: stretch;
}

/* on individual child */
section p {
    justify-self: end;
    align-self: center;
}
\end{minted}

\subsubsection{Span keyword}

\begin{minted}{css}
section p {
    grid-row-start: 1;
    grid-row-end: span(2);
}
\end{minted}

\subsection{Further help}

\href{}{Guide on CSS Tricks}

\begin{infobox}{Firefox}
    Firefox is the best browser for debugging your grid. It has a great deal of tools in it's devtools which are extremely helpful for viewing the grid you have created and it's information.
\end{infobox}

