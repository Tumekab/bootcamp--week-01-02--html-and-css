Floating elements is supposed to be used to move something left or right and have other content move around it. It's used this way in CSS Shapes. However it has been used in the past for layout, before we had flexbox or grid.

\begin{minted}{css}
/* you can float left, right or none */
header h1, header nav {
	float: left;
}
\end{minted}

Floating elements make them jump out of the block scope.
\\
Their containing elements can't see them anymore.
\\
We can get around this by giving the containing (parent) element a \textbf{Block Formatting Context}

This happens when

- is floated
- is positioned absolute
- is displayed inline-block
- has an overflow property
- has a value other than visible


\begin{minted}{css}
header {
	overflow: auto;
	/* OR */
	display: flow-root;
}

header h1, header nav {
	float: left;
}
\end{minted}
