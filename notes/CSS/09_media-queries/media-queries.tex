There are lots and lots of different ways you can consume websites and webapps these days. Monitors, phones, TVs, game consoles and watches to name a few.
\\
We have only been looking at our web pages on a screen so far. We need to make sure they look good at all sizes.
\\
To do this we will use responsive design techniques. This is different from adaptive design which means serving a whole different site or app for different devices.

\begin{infobox}{All environments}
    Responsive design isn't just about making sure things look good at different screen size. It should also take into account the environment the user is in when using your site or app, like how noisy things are or what data connection they have.
\end{infobox}

\subsection{How?}

- Use relative not absolute CSS units
- CSS Media Queries |
- Is also JavaScript feature and device detection |


\subsubsection{Media queries}

Different types of media, like print

\begin{minted}{css}
@media print {

	* {
		background-color: transparent;
	}

}
\end{minted}

Breakpoints (viewport width)

\begin{minted}{css}
/* phone */
@media screen and (max-width: 500px) {

	#left-column {
		display: none;
	}

}
\end{minted}


\begin{minted}{css}
/* Extra small devices (phones, less than 768px) */
@media screen and (max-width: 767px) { ... }

/* Small devices (tablets, 768px and up) */
@media screen and (min-width: 768px) and (max-width: 991px) { ... }

/* Medium devices (desktops, 992px and up) */
@media screen and (min-width: 992px) and (max-width: 1199px) { ... }

/* Large devices (large desktops, 1200px and up) */
@media screen and (min-width: 1200px) { ... }
\end{minted}


\begin{infobox}{Testing}
    You can move the screen width when devtools is open. There is also a mobile/tablet icon at the top, if you select this you can choose a number of screen sizes.
\end{infobox}

\subsection{Loading different pictures}

You can use the \texttt{picture} element to load different image types and also different files based on screen size

\begin{minted}{html}
<picture>
	<source srcset="surfer.png" media="(min-width: 800px)">
	<img src="surfer.jpg" />
</picture>
\end{minted}


\subsubsection{Object fit}

You can use the `object-fit` property to display how the image appears in it's container.

\begin{minted}{css}
img {
	height: 100%; width: 100%;
	display: block;
	object-fit: contain;
}
\end{minted}

\href{https://developer.mozilla.org/en-US/docs/Web/HTML/Element/picture}{Further reading}